% Template for PLoS
% Version 3.4 January 2017
\documentclass[10pt,letterpaper]{article}
\usepackage[top=0.85in,left=2.75in,footskip=0.75in]{geometry}

% amsmath and amssymb packages, useful for mathematical formulas and symbols
\usepackage{amsmath,amssymb}

% Use adjustwidth environment to exceed column width (see example table in text)
\usepackage{changepage}

% Use Unicode characters when possible
\usepackage[utf8x]{inputenc}

% textcomp package and marvosym package for additional characters
\usepackage{textcomp,marvosym}

% cite package, to clean up citations in the main text. Do not remove.
% \usepackage{cite}

% Use nameref to cite supporting information files (see Supporting Information section for more info)
\usepackage{nameref,hyperref}

% line numbers
\usepackage[right]{lineno}

% ligatures disabled
\usepackage{microtype}
\DisableLigatures[f]{encoding = *, family = * }

% color can be used to apply background shading to table cells only
\usepackage[table]{xcolor}

% array package and thick rules for tables
\usepackage{array}

% create "+" rule type for thick vertical lines
\newcolumntype{+}{!{\vrule width 2pt}}

% create \thickcline for thick horizontal lines of variable length
\newlength\savedwidth
\newcommand\thickcline[1]{%
  \noalign{\global\savedwidth\arrayrulewidth\global\arrayrulewidth 2pt}%
  \cline{#1}%
  \noalign{\vskip\arrayrulewidth}%
  \noalign{\global\arrayrulewidth\savedwidth}%
}

% \thickhline command for thick horizontal lines that span the table
\newcommand\thickhline{\noalign{\global\savedwidth\arrayrulewidth\global\arrayrulewidth 2pt}%
\hline
\noalign{\global\arrayrulewidth\savedwidth}}


% Remove comment for double spacing
%\usepackage{setspace}
%\doublespacing

% Text layout
\raggedright
\setlength{\parindent}{0.5cm}
\textwidth 5.25in
\textheight 8.75in

% Bold the 'Figure #' in the caption and separate it from the title/caption with a period
% Captions will be left justified
\usepackage[aboveskip=1pt,labelfont=bf,labelsep=period,justification=raggedright,singlelinecheck=off]{caption}
\renewcommand{\figurename}{Fig}

% Use the PLoS provided BiBTeX style
% \bibliographystyle{plos2015}

% Remove brackets from numbering in List of References
\makeatletter
\renewcommand{\@biblabel}[1]{\quad#1.}
\makeatother

% Leave date blank
\date{}

% Header and Footer with logo
\usepackage{lastpage,fancyhdr,graphicx}
\usepackage{epstopdf}
\pagestyle{myheadings}
\pagestyle{fancy}
\fancyhf{}
\setlength{\headheight}{27.023pt}
\lhead{\includegraphics[width=2.0in]{PLOS-submission.eps}}
\rfoot{\thepage/\pageref{LastPage}}
\renewcommand{\footrule}{\hrule height 2pt \vspace{2mm}}
\fancyheadoffset[L]{2.25in}
\fancyfootoffset[L]{2.25in}
\lfoot{\sf PLOS}

%% Include all macros below
\newcommand{\lorem}{{\bf LOREM}}
\newcommand{\ipsum}{{\bf IPSUM}}





\usepackage{forarray}
\usepackage{xstring}
\newcommand{\getIndex}[2]{
  \ForEach{,}{\IfEq{#1}{\thislevelitem}{\number\thislevelcount\ExitForEach}{}}{#2}
}

\setcounter{secnumdepth}{0}

\newcommand{\getAff}[1]{
  \getIndex{#1}{York University}
}

\providecommand{\tightlist}{%
  \setlength{\itemsep}{0pt}\setlength{\parskip}{0pt}}

\begin{document}
\vspace*{0.2in}

% Title must be 250 characters or less.
\begin{flushleft}
{\Large
\textbf\newline{Simple principles for engineering reproducible solutions to
environmental management challenges.} % Please use "sentence case" for title and headings (capitalize only the first word in a title (or heading), the first word in a subtitle (or subheading), and any proper nouns).
}
\newline
\\
Malory Owen\textsuperscript{\getAff{York University}},
Christopher J. Lortie\textsuperscript{\getAff{Biology, York University}}\textsuperscript{*}\\
\bigskip
\textbf{\getAff{York University}}Biology, 4700 Keele St.~Toronto, ON, Canada, M3J1P3\\
\bigskip
* Corresponding author: lortie@yorku.ca\\
\end{flushleft}
% Please keep the abstract below 300 words
\section*{Abstract}
An environmental management challenge is an opportunity to use
fundamental science to inform evidence-based decisions for environmental
stakeholders and conservationists. Contemporary science is embracing
open science and increasingly conscious of reproduciblility.
Synergistically, applying these two paradigms in concert advances our
capacity to move beyond context dependency and singlular thinking to
reverse engineer solutions from published scientific evidence associated
with one challenge to many. Herein, we provide a short list of
principles that can guide those that seek solutions to address
environmental management through primary scientific literature.

% Please keep the Author Summary between 150 and 200 words
% Use first person. PLOS ONE authors please skip this step.
% Author Summary not valid for PLOS ONE submissions.
\section*{Author summary}
Grand challenges require grand solutions. Environmental management
cannot neglect fundamental science as a substrate for effective decision
making, and scientists should be conscious of how their science can be
used by managers.

\linenumbers

% Use "Eq" instead of "Equation" for equation citations.
\section{Introduction}\label{introduction}

Conservation decisions can reside with legislators (indirect impactors)
or environmental managers (direct impactors). To focus on the latter,
managers typically have scientific backgrounds and routinely navigate
the technical literature. However, engagement with scientific literature
is non-trivial for all practitioners because of time, access, relevance
of the science, and reporting standards. Environmental managers and
conservationist are certainly seeking evidence to inform decisions
(citations please). However, there can be a gap in communication between
basic science and management for at least three reasons. Firstly, the
research is not a direct study of an ecosystem, and an immediate,
real-world solution is needed by managers - preferably with a
demonstrable outcome and reasonable cost estimate (citations). This is a
very real limitation in the primary science literature restoration
ecology for instance (cite Lortie better late than never). Secondly, the
link between the biology or ecology present in the literature is not
articulately connected to the similar process for the system at hand.
There are notable examples with journals just as the Journal of Applied
Ecology, Basic and Applied Ecology, the Journal of Environmental
Engineering, and many others, but there nonetheless remains a vast pool
of opportunity for solution development in other journals.
Context-specific findings in science are a legimitate and useful means
to advance discovery, but at times, studies from one system can be
repurposed for insights into another. Finally, the capacity to see the
forest for the trees or the generality of a specific, even large-scale
or broad basic research study can be a challenge. Science can be very
specialized (citation to Nancy Baron book), and mobilizing knowledge for
solutions requires both detailed expertise, scientific synthesis tools
(cite to Lortie formalized synthesis paper in Oikos), or a focus on
identifying the salient elements associated with a study (citation).
Often, seeing the forest also requires sampling many trees. This leads
to the general proposal here that experts can promote reproducible and
mobile knowledge for many fundamental science endeavors by considering
these principles. This is both a set of principles for how to make your
research potentially reusable by environmental managers and
conservationists (citation to get your paper used by meta) and inform
solutions for the environmental crisis.

An environmental management challenge is a problem presented in
scientific literature or society that, when redefined and reviewed using
these 10 principles, can result in a solution to the original problem.
EXPAND\ldots{} a tiny bit more..how. Typically, a challenge is\ldots{}.
look up how they define grand challenges and make mention here including
the 17 global sustainability goals. A (reverse-engineered) reproducible
solution is a suggested solution to a challenge derived from identifying
all the components of the challenge. It should be applicable to multiple
local-extent challenges when tweaked to fit the circumstances. Finally,
a tool or solution is the desired outcome from the primary research to
support evidence-based decision making in conservation. In this case, a
tool is a methodology researchers use that can faciitate managers to
either identify best ways to measure/identify issues or to provide
solutions for their specific challenge. Any tool is linked to its
respective reproducible solution by the fundamental concepts of
reproducibility (citation to reproducibility crisis paper). These can
include primarily conceptual replication but also\ldots{}. check papers
on reproducibility. then concluding sentence.

The heuristic developed here was inspired by the ``Ten Simple Rules''
format pioneered by Phillip Bourne in the field of computational biology
{[}1{]}. We propose that by distilling the concepts that promote
engagement with scientific literature outside of the research community,
managers can rely on broader sources of scientific knowledge to make
decisions. Additionally, researchers can better understand the
perspective of managers facilitating science and scientific
communication that is more applicable to managers without compromising
their research programs. Here, we will outline and discuss simple
``principles'' scientists can use to make their research more applicable
to managers, and managers can use to identify basic science that fits
their needs.

\section{The principles}\label{the-principles}

\textbf{1. Reframe the problem as challenge.} Doom-and-gloom is a
pervasive mentality in ecology and environmental sciences which can
reduce productivity and problem solving. It can shut down even the most
motivated of minds--but beyond the issue of motivation, reframing a
problem as a \emph{challenge} can reveal solution sets that may
otherwise remain hidden. For example, consider the problem of
human-wildlife conflict between carnivores and the people living near
the Ruaha National Park boundary in Tanzania. The \emph{problem} is that
98.5\% of people perceive wildlife as a threat to their livestock
resulting in increased likelihoods for human-wildlife conflict {[}2{]}.
Reframed, the challenge can be to decrease the negative perception of
wildlife in areas with high human-wildlife conflict opportunities. It is
a small change in semantics, but it is a potentially profound change in
direction.

\textbf{2. Describe the scope and extent of the challenge.} Defining the
scope of a challenge conceptually and the extent geographically will
ensure that potential solutions fit the challenge. Moving across scales
is a common issue in ecology (citation to Sandel 2015 paper in
Ecography), and proposing a spatial scale, using common terms, and
describing the breadth of the challenge will accelerate
interdisciplinary solutions (i.e.~the wildlife-human challenge above is
ecological and societal). The challenge can be problematic on multiple
scales including local, regional, or global scale and solutions can be
needed for each. Conceptually, the scope is broad in the human-wildlife
conflict example whilst the extent is primarily local to the area
surrounding the Southern border of the Ruaha National Park.

\textbf{3. Explicitly link the basic science to management implications
and policy.} Perhaps the most facile principle, a simple description and
definition of the basic scientific evidence in a study and how it can be
resued is a fundamental step in linking science to evidence-based
decision making for environmental challenges. In the wildlife-human
challenge, depredation of livestock impacted 61.1\% of households, but
livestock loss due to disease or theft was X percent? of total loss
{[}2{]}. Consequently, a clear statement of evidence can illuminate the
most viable solution sets in some instances.

\textbf{4. Propose implications of ignoring this challenge.} A
description of the impact a challenge on a system if left unchecked will
help clarify the severity of the challenge. The trickle down effects and
indirect implications of the challenge should also be examined. For
instance, anti-carnivore sentiment will likely only grow as climate
change and pressures to confine pastoral herders makes livestock more
difficult to raise {[}3,4{]}. Many large carnivores are already
threatened/endangered, and further anthropogenic pressures on the
populations will lead to severe declines in populations including
potential extinction of keystone species {[}5--7{]}; but it is often
associated with underlying human-human conflict {[}8{]}.

\textbf{5. State the direct human needs associated with this challenge.}
State the direct needs of humans as part of the process of generating
reproducible solutions for environmental challenges. The intrinsic value
of the ecosystem is unrefutable (citation) but also linking the
challenge and its solutions to direct human needs makes it less likely
to be dismissed. Identifying anthropogenic needs will help a problem
solver create a solution that is appropriate for the challenge, and it
can also prevent the emergence of new related challenges or pressures on
the system in question. This principle can also include engagement with
stakeholders as a mechanism to inform benefits and solutions (citations
to colvin and reed papers on github issue).

\textbf{6. List at least one limitation of the study and explain.} There
is no perfect experiment (Ruxton citation) or synthesis (Kotiaho \&
Tomkins citation). Critically reading the study associated with the
challenge can mean the difference between success and failure of a later
implemented management solution that otherwise follows all other
principles presented here. A clear analysis of causation and correlation
can help avoid a fatal misstep and ensures effective framing of expected
outcomes with an environmental intervention for consevationists. This is
not to say that interventions need always be cause-effect studies or
that evidence-based decisions cannot be made with compelling preliminary
evidence or mensurative data. We are simply proposing that a statement
of the relative strength of evidence and gaps in the research provides a
future direction for additional research and for implementation.

\textbf{7. Explore the benefits of minimal intervention for
stakeholders.} Resources are limiting, and at times, the
business-as-usual model can provide a guide to intervention for some
environmental management challenges (citation to a BAU for environmental
man paper). At the minimum, exploration of a hope for the best strategy
or minimal intervention is critical because of costs. Business-as-usual
models can also provide an economic mechanism to value ecosystems
services (citation) and whilst this is not withour debate, this can
expand the breath of stakeholders and potential investors in a solution
for a particular challenge. A best and worst case scenario analysis is
also likely a frequent need for many environmental challenges because of
intertia in the structures that we use to manage people and resources.

\textbf{8. List the tools applied to this challenge.} In an
environmental management challenge case study, there is typically at
least one primary tool that the researchers used to explore a challenge,
and there are many (Table 1). The tools in basic biology and ecology
relevant to environmental management can be reproducible it at least
conceptually, they can be replicated in another system or applied to
similar challenge - i.e.~citizen science as a means to collect
environmental data (citation) is relevant to many of the challenges we
face including global warming, water quality, and declining
biodiversity.

also - put table at the end before lit cited.

I would use a quick r-code chunk and use knitr::kable like I did for
tables in bio4enviro - see webpage. look great and easy. just use
include = FALSE and it will show only output.

Table. 1. - look up up to make a table in rmd. easy peasy .and will look
better. \emph{meta-analysis {[}9{]}; {[}10{]} }systematic review
{[}11{]} \emph{citizen science {[}12{]}; {[}13{]} }team science {[}14{]}
\emph{R {[}15{]} }mapping {[}16{]} \emph{big data/open access databases
{[}17{]}; {[}18{]}; {[}19{]} }modelling {[}20{]} \emph{surveys {[}21{]}
}biomonitoring {[}22{]} \emph{evolutionary change/population viability
genetic analysis {[}23{]} }economic incentivization monitoring {[}24{]}
*human health and well-being monitoring {[}25{]}; {[}2{]} This list
could potentially be expanded upon, and new tools can result in new
solutions.

\textbf{9. Link the primary reproducible tool to the outcome.} The tool
can provide a means to collect data, detect patterns, directly solve an
environmental challenge, or inform policy. If the paper was a direct
test of basic ecology for an environmental challenge, this can be very
straightforward. For instance, ``Odonata (Insecta) as a tool for the
bio-monitoring of environmental quality'' {[}22{]} clearly provides a
means to measure and detect. However, the other proposed roles can
address challenges in a diversity of ways. The identification of or
provision of research evidence is the most ``basic'' role, and it is
also likely the most typical in ecology. Tools that can function in this
capacity include surveys, meta-analyses, systematic reviews, citizen
science data collection, mapping, open-access data, modelling to predict
changes. Tests in the second category directly examine the efficacy of a
management strategy or intervention. Tools that could play this role
include bio-monitoring, mitigation and remediation experiments, and
population demography studies. The inform policy research is typically
more indirect and synthetic. They often take the form of anthropocentric
studies that consider ecological or environmental policy. Any of the
above toosl can serve this role, but some tools fit most squarely in
this role include economic incentivization models (citation), human
health impact studies (citation), and human well-being monitoring
associated with environmental interventions (citation to the Weds paper
this week on evidence mapping).

\textbf{10. Apply the tool to another challenge or explain how it is
generalizable.} This principle proposes that the primary tool is
reproducible if it can be applied to another challenge or context. It
ties together the concept that reverse-engineered reproducible solutions
are relevant to more than the unpacked, single environmental management
challenge case. This can promote increased in efficiency for tackling
novel environmental challenges as they emerge, and it also supports the
overaraching philosophy here for basic science that we cannot continue
to ignore reuse given the global environmental needs for better decision
making.

\section{Implications}\label{implications}

At its core, these principles can distribute the burden of scientific
communication between scientists and stakeholders and embodies a spirit
of dialog between senders and receivers (or between producers and
consumers). Reuse is also not the sole criterion for useful science nor
should it be. Professional advocacy and knowledge mobilization are
increasingly important priorities for universities and science in
general (citations). Evidence-based decision making is a critical area
for growth and knowledge in many disciplines (citation) - not just
environmental management. Increased consumption of scientific evidence
by managers and basic science that is more palatable to a broader
audience written by researchers will result in increased functional use
of scientific literature. Collaboration with stakeholders will facilate
this process, and open science will be pivotal.

\section*{References}\label{references}
\addcontentsline{toc}{section}{References}

\hypertarget{refs}{}
\hypertarget{ref-Bourne2006}{}
1. Bourne PE, Chalupa LM. Ten simple rules for getting grants. PLoS
Computational Biology. 2006;2: 59--60.
doi:\href{https://doi.org/10.1371/journal.pcbi.0020012}{10.1371/journal.pcbi.0020012}

\hypertarget{ref-Dickman2014}{}
2. Dickman AJ, Hazzah L, Carbone C, Durant SM. Carnivores, culture and
'contagious conflict': Multiple factors influence perceived problems
with carnivores in Tanzania's Ruaha landscape. Biological Conservation.
Elsevier Ltd; 2014;178: 19--27.
doi:\href{https://doi.org/10.1016/j.biocon.2014.07.011}{10.1016/j.biocon.2014.07.011}

\hypertarget{ref-Jones2009}{}
3. Jones PG, Thornton PK. Croppers to livestock keepers: livelihood
transitions to 2050 in Africa due to climate change. Environmental
Science and Policy. 2009;12: 427--437.
doi:\href{https://doi.org/10.1016/j.envsci.2008.08.006}{10.1016/j.envsci.2008.08.006}

\hypertarget{ref-Lindsey2009}{}
4. Lindsey PA, Romañach SS, Davies-Mostert HT. The importance of
conservancies for enhancing the value of game ranch land for large
mammal conservation in southern Africa. Journal of Zoology. 2009;277:
99--105.
doi:\href{https://doi.org/10.1111/j.1469-7998.2008.00529.x}{10.1111/j.1469-7998.2008.00529.x}

\hypertarget{ref-Towns2009}{}
5. Towns L, Derocher AE, Stirling I, Lunn NJ, Hedman D. Spatial and
temporal patterns of problem polar bears in Churchill, Manitoba. Polar
Biology. 2009;32: 1529--1537.
doi:\href{https://doi.org/10.1007/s00300-009-0653-y}{10.1007/s00300-009-0653-y}

\hypertarget{ref-Bagchi2006}{}
6. Bagchi S, Mishra C. Living with large carnivores: Predation on
livestock by the snow leopard (Uncia uncia). Journal of Zoology.
2006;268: 217--224.
doi:\href{https://doi.org/10.1111/j.1469-7998.2005.00030.x}{10.1111/j.1469-7998.2005.00030.x}

\hypertarget{ref-Johnson2006}{}
7. Johnson A, Vongkhamheng C, Hedemark M, Saithongdam T. Effects of
human-carnivore conflict on tiger (Panthera tigris) and prey populations
in Lao PDR. Animal Conservation. 2006;9: 421--430.
doi:\href{https://doi.org/10.1111/j.1469-1795.2006.00049.x}{10.1111/j.1469-1795.2006.00049.x}

\hypertarget{ref-Dickman2010}{}
8. Dickman AJ. Complexities of conflict: The importance of considering
social factors for effectively resolving human-wildlife conflict. Animal
Conservation. 2010;13: 458--466.
doi:\href{https://doi.org/10.1111/j.1469-1795.2010.00368.x}{10.1111/j.1469-1795.2010.00368.x}

\hypertarget{ref-Castanho2015}{}
9. Castanho C de T, Lortie CJ, Zaitchik B, Prado PI. A meta-analysis of
plant facilitation in coastal dune systems: responses, regions, and
research gaps. PeerJ. 2015;3: e768.
doi:\href{https://doi.org/10.7717/peerj.768}{10.7717/peerj.768}

\hypertarget{ref-Busch2017}{}
10. Busch J, Ferretti-Gallon K. What drives deforestation and what stops
it? A meta-analysis. Review of Environmental Economics and Policy.
2017;11: 3--23.
doi:\href{https://doi.org/10.1093/reep/rew013}{10.1093/reep/rew013}

\hypertarget{ref-Lortie2018}{}
11. Lortie CJ, Filazzola A, Kelsey R, Hart AK, Butterfield HS. Better
late than never: a synthesis of strategic land retirement and
restoration in California. Ecosphere. 2018;9: e02367.
doi:\href{https://doi.org/10.1002/ecs2.2367}{10.1002/ecs2.2367}

\hypertarget{ref-Burkle2013}{}
12. Burkle LA, Marlin JC, Knight TM. Plant-Pollinator Interactions over
120 Years: Loss of Spcies, Co-Occurence, and Function. Science.
2013;339: 1611--1616.

\hypertarget{ref-Conrad2011}{}
13. Conrad CC, Hilchey KG. A review of citizen science and
community-based environmental monitoring: Issues and opportunities.
Environmental Monitoring and Assessment. 2011;176: 273--291.
doi:\href{https://doi.org/10.1007/s10661-010-1582-5}{10.1007/s10661-010-1582-5}

\hypertarget{ref-Nielsen2017}{}
14. Nielsen JA, Grøndahl E, Callaway RM, Dickinson KJ, Ehlers BK. Home
and away: biogeographical comparison of species diversity in Thymus
vulgaris communities. Biological Invasions. Springer International
Publishing; 2017;19: 2533--2542.
doi:\href{https://doi.org/10.1007/s10530-017-1461-x}{10.1007/s10530-017-1461-x}

\hypertarget{ref-McCarthy2010}{}
15. McCarthy MP, Best MJ, Betts RA. Climate change in cities due to
global warming and urban effects. Geophysical Research Letters. 2010;37:
1--5.
doi:\href{https://doi.org/10.1029/2010GL042845}{10.1029/2010GL042845}

\hypertarget{ref-Halpern2008}{}
16. Halpern B, Walbridge S, Selkoe K, Kappel C, Micheli F, D'Argrosa C,
et al. A Global Map of Human Impact on Marine Ecosystems. Science.
2008;319: 948--953.
doi:\href{https://doi.org/10.1111/2041-210X.12109}{10.1111/2041-210X.12109}

\hypertarget{ref-Sillero2014}{}
17. Sillero N, Campos J, Bonardi A, Corti C, Creemers R, Crochet PA, et
al. Updated distribution and biogeography of amphibians and reptiles of
Europe. Amphibia Reptilia. 2014;35: 1--31.
doi:\href{https://doi.org/10.1163/15685381-00002935}{10.1163/15685381-00002935}

\hypertarget{ref-Dengler2011}{}
18. Dengler J, Jansen F, Glöckler F, Peet RK, Cáceres M de, Chytrý M, et
al. The Global Index of Vegetation-Plot Databases (GIVD): A new resource
for vegetation science. Journal of Vegetation Science. 2011;22:
582--597.
doi:\href{https://doi.org/10.1111/j.1654-1103.2011.01265.x}{10.1111/j.1654-1103.2011.01265.x}

\hypertarget{ref-Maldonado2015}{}
19. Maldonado C, Molina CI, Zizka A, Persson C, Taylor CM, Albán J, et
al. Estimating species diversity and distribution in the era of Big
Data: To what extent can we trust public databases? Global Ecology and
Biogeography. 2015;24: 973--984.
doi:\href{https://doi.org/10.1111/geb.12326}{10.1111/geb.12326}

\hypertarget{ref-Vogt2018}{}
20. Vogt R, Sharma S, Leavitt P. Direct and interactive effects of
climate, meteorology, river hydrology, and lake characteristics on water
quality in productive lakes of the Canadian Prairies. Canadian Journal
of Fisheries and Aquatic Sciences. 2018;75: 47--59.
doi:\href{https://doi.org/10.1139/cjfas-2016-0520}{10.1139/cjfas-2016-0520}

\hypertarget{ref-Wassen2005}{}
21. Wassen MJ, Venterink HO, Lapshina ED, Tanneberger F. Endangered
plants persist under phosphorus limitation. Nature. 2005;437: 547--550.
doi:\href{https://doi.org/10.1038/nature03950}{10.1038/nature03950}

\hypertarget{ref-Miguel2017}{}
22. Miguel TB, Oliveira-Junior JMB, Ligeiro R, Juen L. Odonata (Insecta)
as a tool for the biomonitoring of environmental quality. Ecological
Indicators. Elsevier; 2017;81: 555--566.
doi:\href{https://doi.org/10.1016/j.ecolind.2017.06.010}{10.1016/j.ecolind.2017.06.010}

\hypertarget{ref-Stoops2016}{}
23. Stoops MA, Campbell MK, DeChant CJ, Hauser J, Kottwitz J, Pairan RD,
et al. Enhancing captive Indian rhinoceros genetics via artificial
insemination of cryopreserved sperm. Animal Reproduction Science.
Elsevier B.V. 2016;172: 60--75.
doi:\href{https://doi.org/10.1016/j.anireprosci.2016.07.003}{10.1016/j.anireprosci.2016.07.003}

\hypertarget{ref-Cerda2018}{}
24. Cerda C, Fuentes JP, De La Maza CL, Louit C, Araos A. Assessing
visitors' preferences for ecosystem features in a desert biodiversity
hotspot. Environmental Conservation. 2018;45: 75--82.
doi:\href{https://doi.org/10.1017/S0376892917000200}{10.1017/S0376892917000200}

\hypertarget{ref-Mergler2007}{}
25. Mergler D, Anderson HA, Chan LHM, Mahaffey KR, Murray M, Sakamoto M,
et al. Methylmercury exposure and health effects in humans: A worldwide
concern. Ambio. 2007;36: 3--11.
doi:\href{https://doi.org/10.1579/0044-7447(2007)36\%5B3:MEAHEI\%5D2.0.CO;2}{10.1579/0044-7447(2007)36{[}3:MEAHEI{]}2.0.CO;2}

\nolinenumbers


\end{document}

