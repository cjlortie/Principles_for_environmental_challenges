%% Submissions for peer-review must enable line-numbering
%% using the lineno option in the \documentclass command.
%%
%% Preprints and camera-ready submissions do not need
%% line numbers, and should have this option removed.
%%
%% Please note that the line numbering option requires
%% version 1.1 or newer of the wlpeerj.cls file, and
%% the corresponding author info requires v1.2

\documentclass[fleqn,10pt]{wlpeerj} % for preprint submissions

% ZNK -- Adding headers for pandoc

\setlength{\emergencystretch}{3em}
\providecommand{\tightlist}{
\setlength{\itemsep}{0pt}\setlength{\parskip}{0pt}}
\usepackage{lipsum}
\usepackage[unicode=true]{hyperref}
\usepackage{longtable}



\usepackage{lipsum}

\title{Simple principles for engineering reproducible solutions to
environmental management challenges.}

\author[1]{Christopher J. Lortie}

\corrauthor[1]{Christopher J. Lortie}{\href{mailto:chris@christopherlortie.info}{\nolinkurl{chris@christopherlortie.info}}}
\author[1]{Malory Owen}


\affil[1]{Biology, York University, 4700 Keele St.~Toronto, ON, Canada, M3J1P3}


%
% \author[1]{First Author}
% \author[2]{Second Author}
% \affil[1]{Address of first author}
% \affil[2]{Address of second author}
% \corrauthor[1]{First Author}{f.author@email.com}

% 

\begin{abstract}
An environmental management challenge is an opportunity to use
fundamental science to inform evidence-based decisions for environmental
stakeholders and conservationists. Contemporary science is embracing
open science and increasingly conscious of reproduciblility.
Synergistically, applying these two paradigms in concert advances our
capacity to move beyond context dependency and singlular thinking to
reverse engineer solutions from published scientific evidence associated
with one challenge to many. Herein, we provide a short list of
principles that can guide those that seek solutions to address
environmental management through primary scientific literature.
% Dummy abstract text. Dummy abstract text. Dummy abstract text. Dummy abstract text. Dummy abstract text. Dummy abstract text. Dummy abstract text. Dummy abstract text. Dummy abstract text. Dummy abstract text. Dummy abstract text.
\end{abstract}

\begin{document}

\flushbottom
\maketitle
\thispagestyle{empty}

\section*{Introduction}\label{introduction}
\addcontentsline{toc}{section}{Introduction}

Conservation decisions can reside with legislators or with environmental
managers. To focus on the latter, managers typically have scientific
backgrounds and routinely navigate the technical literature. However,
engagement with scientific literature is non-trivial for all
practitioners because of time, access, relevance of the science, and
reporting standards. Environmental managers and conservationists need to
be able to use evidence to inform decisions (Koontz and Thomas 2018;
Cash et al. 2003). However, there can be a gap in communication between
basic science and management for at least three reasons. Firstly, the
research is not a direct study of an ecosystem, and an immediate,
real-world solution is needed by managers - preferably with a
demonstrable outcome and reasonable cost estimate (Iacona et al. 2018;
Naidoo et al. 2006). This is a very real limitation in the primary
science literature restoration ecology for instance (Lortie et al.
2018). Secondly, the link between the biology or ecology present in the
literature is not articulately connected to the similar process for the
system at hand. There are notable examples with journals just as the
Journal of Applied Ecology, Basic and Applied Ecology, the Journal of
Environmental Engineering, and many others. Nonetheless, there remains a
vast pool of opportunity for solution development from publications in
other journals. Context-specific findings in science are a legitimate
and useful means to advance discovery, but at times, studies from one
system can be re-purposed for insights into another. Finally, the
capacity to see the forest for the trees for even large-scale or broad
basic research study can be a challenge. Science can be very specialized
(Baron 2010), and mobilizing knowledge for solutions requires both
detailed expertise, scientific synthesis tools (Lortie 2014), or a focus
on identifying the salient elements associated with a study (Lewinsohn
et al. 2014; Hao 2018). Often, seeing the forest also requires sampling
many trees. This leads to the general proposal here that experts can
promote reproducible and mobile knowledge for many fundamental
scientific endeavors by considering these principles. This is both a set
of principles for how to make your research potentially reusable by
environmental managers and conservationists (Gerstner et al. 2017) and
inform solutions for the environmental crisis.

An environmental management challenge is a problem presented in
scientific literature or society that, when redefined and reviewed using
these principles, can result in a solution to the original problem.
Typically, a challenge is ethical, legal, social, or derived from
implications associated with research and evidence of change or
anthropogenic stress (Acocella 2015). Grand challenges for the
environment in particular are ones that necessitate connections between
disciplines and require evidence from potential studies that examine
different components of the environment such as climate, ecology,
species biology, or research from any number of levels (Macpherson and
Segarra 2017). A (reverse-engineered) reproducible solution is a
suggested solution to a challenge derived from identifying all the
components of the challenge. In software engineering, this process
includes analysis of the architecture of a system, examining the
relationships between subsystems, and creating a mental model of how the
system functions (Fiutem and Antoniol, n.d.). The same process can be
applied to basic science as a system for supporting environmental
management decisions. It should be applicable to multiple local-extent
challenges when adjusted to fit the circumstances (like a software
application that can run under different operating systems). Finally, a
tool or solution is the desired outcome from the primary research to
support evidence-based decision making in conservation. In this case, a
tool is a methodology researchers use that can facilitate managers to
either identify best ways to measure/identify issues or to provide
solutions for their specific challenge. Any tool is linked to its
respective reproducible solution by the fundamental concepts of
reproducibility (Baker 2016). These can include primarily conceptual
replication, i.e.~repeating the ideas, but there are many other
solutions. Here, we propose that both direct replication (replicating
the same approach in another context) and conceptual replication
(repeated tests of the same concept but with different methods) (Kelly
2006) will advance our capacity to explore reproducibility of basic
science to different challenges associated with environmental
management. The primary goal is to escape the `everything is
context-specific' assumption sometimes applied to many natural science
subdisciplines.

The heuristic developed here was inspired by the `ten simple rules'
paper format pioneered by Phillip Bourne in the field of computational
biology (Bourne and Chalupa 2006). We propose that by distilling the
concepts that promote engagement with scientific literature outside of
the research community, managers can rely on broader sources of
scientific knowledge to make decisions. Additionally, researchers can
better understand the perspective of managers facilitating science and
scientific communication that is more relevant to managers without
compromising their research respective programs. Here, we will outline
and discuss simple ``principles'' scientists can use to make their
research more applicable to managers and that managers can in turn use
to identify basic science that fits their needs.

\section*{The principles}\label{the-principles}
\addcontentsline{toc}{section}{The principles}

\textbf{1. Reframe the problem as challenge.} Doom-and-gloom is a
pervasive theme in the reporting and social media discussions of ecology
and environmental sciences that can reduce our productivity and capacity
to solve problems. It can shut down even the most motivated of minds --
but beyond the issue of motivation, reframing a problem as a
\emph{challenge} can reveal solution sets that otherwise remain hidden.
For example, consider the problem of human-wildlife conflict between
carnivores and the people living near the Ruaha National Park boundary
in Tanzania. The \emph{problem} is that 98.5\% of people perceive
wildlife as a threat to their livestock resulting in increased
likelihood for human-wildlife conflict (Amy J. Dickman et al. 2014).
Re-framed, the challenge can be to improve perception of wildlife in
areas with high human-wildlife conflict opportunities. It is a small
change in semantics but a potentially profound change in direction. The
challenge can also include improving experiences for people with
wildlife or reducing their losses to wildlife.

\textbf{2. Describe the scope and extent of the challenge.} Defining the
scope of a challenge conceptually and the extent geographically will
ensure that potential solutions fit the challenge. Moving across scales
is a common issue in ecology (Sandel 2015), and proposing a spatial
scale, using common terms, and describing the breadth of the challenge
will accelerate interdisciplinary solutions (i.e.~the wildlife-human
challenge above is ecological and societal). The challenge can be
problematic on local, regional, or global scales, and solutions can be
needed for each. Conceptually, the scope is broad in the human-wildlife
conflict example whilst the extent is primarily local to the area
surrounding the Southern border of the Ruaha National Park. Articulating
scope and scale informs assessment of severity.

\textbf{3. Explicitly link the basic science to management implications
and policy.} Perhaps the most facile principle, a simple description and
definition of the basic scientific evidence in a study and how it can be
reused is a fundamental step in linking science to evidence-based
decision making for environmental challenges. In the wildlife-human
challenge, depredation of livestock impacted 61.1\% of households in
some form, but livestock losses due to disease or theft were actually
the most consistent negative drivers of total loss (Amy J. Dickman et
al. 2014). Perception of loss and actual losses were not necessarily
equivalent, and culture was shaping subsequent conflicts not direct
evidence. Consequently, a clear statement of evidence can illuminate the
most viable solution sets in some instances.

\textbf{4. Propose implications of ignoring this challenge.} A
description of the impact a challenge on a system if left unchecked will
help clarify the severity of the challenge. The trickle down effects and
indirect implications of the challenge should also be examined. For
instance, anti-carnivore sentiment will likely only grow as climate
change and pressures to confine pastoral herders makes livestock more
difficult to raise (P. G. Jones and Thornton 2009; Lindsey, Romañach,
and Davies-Mostert 2009). Many large carnivores are already threatened
and endangered, and further anthropogenic pressures on the populations
will lead to severe declines in populations including potential
extinction of keystone species (Towns et al. 2009; Bagchi and Mishra
2006; Johnson et al. 2006); but it is often associated with underlying
human-human conflict (A. J. Dickman 2010).

\textbf{5. State the direct human needs associated with this challenge.}
State the direct needs of humans as part of the process of generating
reproducible solutions for environmental challenges. The intrinsic value
of the ecosystem is impossible to quantify (Davidson 2013), but linking
the challenge and its solutions to direct human needs makes it less
likely to be dismissed. Identifying anthropogenic needs will help a
problem solver create a solution that is appropriate for the challenge,
and it can also prevent the emergence of new related challenges or
pressures on the system in question. This principle can also include
engagement with stakeholders as a mechanism to inform benefits and
solutions (Reed 2008; Colvin, Witt, and Lacey 2016).

\textbf{6. List at least one limitation of the study and explain.} There
is no perfect experiment (Ruxton and Colgrave 2017) or synthesis
(Kotiaho and Tomkins 2002). Critically reading the study associated with
the challenge can mean the difference between success and failure of a
later implemented management solution that otherwise follows all other
principles presented here. A clear analysis of causation and correlation
can help avoid a fatal misstep and ensures effective framing of expected
outcomes with an environmental intervention for conservationists. This
is not to say that interventions need always be cause-effect studies or
that evidence-based decisions cannot be made with compelling preliminary
evidence or mensurative data. We are simply proposing that a statement
of the relative strength of evidence and gaps in the research provides a
future direction for additional research and for implementation.

\textbf{7. Explore the benefits of minimal intervention for
stakeholders.} Resources are limiting, and at times, the
business-as-usual model can provide a guide to intervention for some
environmental management challenges (Ferguson 2015; Mosnier et al.
2017). At the minimum, exploration of a hope-for-the-best strategy or
minimal intervention is critical because of costs. Business-as-usual
models can also provide an economic mechanism to value ecosystems
services (Fu et al. 2018; Karttunen et al. 2018), and whilst this is not
without debate, this can expand the breadth of stakeholders and
potential investors in a solution for a particular challenge. A best and
worst case scenario analysis is also likely a frequent need for many
environmental challenges because of inertia in the socio-political
structures that we use to manage people and resources.

\textbf{8. List the tools applied to this challenge.} In an
environmental management challenge case study, there is typically at
least one primary tool that the researchers used to explore a challenge,
but there are many tools such as meta-analyses (Busch and
Ferretti-Gallon 2017), big data (S. Hampton et al. 2013), mapping
(Halpern et al. 2008), modelling (Vogt, Sharma, and Leavitt 2018),
citizen science (Burkle, Marlin, and Knight 2013), and team science
(Nielsen et al. 2017). The tools in basic biology and ecology relevant
to environmental management can be reproducible if, at least
conceptually, they can be replicated in another system or applied to
similar challenge -- i.e.~citizen science as a means to collect
environmental data (McKinley et al. 2017) is relevant to many of the
challenges we face including global warming, water quality, and
declining biodiversity.

\textbf{9. Link the primary reproducible tool to the outcome.} A
reproducible science tool can provide a means to collect data, detect
patterns, directly solve an environmental challenge, or inform policy.
If the paper was a direct test of basic ecology for an environmental
challenge, this can be very straightforward. For instance, the paper
entitled ``Odonata (Insecta) as a tool for the bio-monitoring of
environmental quality'' (Miguel et al. 2017) clearly provides a means to
measure and detect. However, the other proposed roles can address
challenges in a diversity of ways. The identification of or provision of
research evidence is the most `basic' role, and it is also likely the
most typical role for much of ecology for example. Tools that can
function in this capacity include surveys, citizen science data
collection, mapping, open-access data, and modelling to predict changes.
Tests in the second category that directly examine the efficacy of a
management strategy or intervention can further include bio-monitoring
(Miguel et al. 2017), mitigation and remediation experiments (Zhu, Lu,
and Zhang 2010), and population demography studies (Botero et al. 2015).
Studies that inform policy are typically more indirect and synthetic and
can take the form of anthropocentric studies that consider ecological or
environmental policy. Any of the above tools can serve this role, but
some tools that fit most squarely include economic incentivization
models (Tilman, Levin, and Watson (2018)), human health impact studies
(Chiabai et al. 2018), and human well-being monitoring associated with
environmental interventions (McKinnon 2015).

\textbf{10. Apply the tool to another challenge or explain how it is
generalizable.} This principle proposes that the primary tool is
reproducible if it can be applied to another challenge or context. It
ties together the concept that reverse-engineered reproducible solutions
are relevant to more than the unpacked, single environmental management
challenge case. This can promote increased in efficiency for tackling
novel environmental challenges as they emerge, and it also supports the
overarching philosophy here for basic science that we cannot continue to
ignore reuse given the global environmental needs for better decision
making.

\section*{Implications}\label{implications}
\addcontentsline{toc}{section}{Implications}

These principles can distribute the burden of scientific communication
between scientists and stakeholders and embodies a spirit of dialog
between senders and receivers (or between producers and consumers).
Reuse is also not the sole criterion for useful science nor should it
be, but professional advocacy and knowledge mobilization are
increasingly important priorities for universities and science in
general (Pace et al. 2010). Evidence-based decision making is a critical
area for growth and knowledge in many disciplines (Roy-Byrne et al.
2010; Tranfield, Denyer, and Smart 2003; Aarons, Hurlburt, and Horwitz
2011) -- not just environmental management. Increased consumption of
scientific evidence by managers and basic science that is more palatable
to a broader audience written by researchers will result in increased
functional use of scientific literature. Collaboration with stakeholders
will facilitate this process, and open science will be pivotal. Using at
least some of these principles in the discussion sections of the primary
research we report will increase the stickiness of our ideas and enable
better connections between evidence and outcome, challenge and solution,
and people and nature.

\section*{Acknowledgments}\label{acknowledgments}
\addcontentsline{toc}{section}{Acknowledgments}

This synthesis was supported by NSERC DG to CJL and FGS funding to MO
from York University.

\section*{References}\label{references}
\addcontentsline{toc}{section}{References}

\hypertarget{refs}{}
\hypertarget{ref-Aarons2011}{}
Aarons, Gregory A., Michael Hurlburt, and Sarah Mc Cue Horwitz. 2011.
``Advancing a conceptual model of evidence-based practice implementation
in public service sectors.'' \emph{Administration and Policy in Mental
Health and Mental Health Services Research} 38 (1): 4--23.
doi:\href{https://doi.org/10.1007/s10488-010-0327-7}{10.1007/s10488-010-0327-7}.

\hypertarget{ref-Acocella2015}{}
Acocella, Valerio. 2015. ``Grand Challenges in Earth Science: Research
Toward a Sustainable Environment.'' Journal Article. \emph{Frontiers in
Earth Science} 3: 68.
\url{https://www.frontiersin.org/article/10.3389/feart.2015.00068}.

\hypertarget{ref-Bagchi2006}{}
Bagchi, S., and C. Mishra. 2006. ``Living with large carnivores:
Predation on livestock by the snow leopard (Uncia uncia).''
\emph{Journal of Zoology} 268 (3): 217--24.
doi:\href{https://doi.org/10.1111/j.1469-7998.2005.00030.x}{10.1111/j.1469-7998.2005.00030.x}.

\hypertarget{ref-Baker2016}{}
Baker, M. 2016. ``Is There a Reproducibility Crisis?'' Journal Article.
\emph{Nature} 533: 452--54.

\hypertarget{ref-Baron2010}{}
Baron, N. 2010. \emph{Escape from the Ivory Tower: A Guide to Making
Your Science Matter}. Book. Washington, DC: Island Press.

\hypertarget{ref-Botero2015}{}
Botero, Carlos A., Franz J. Weissing, Jonathan Wright, and Dustin R.
Rubenstein. 2015. ``Evolutionary Tipping Points in the Capacity to Adapt
to Environmental Change.'' Journal Article. \emph{Proceedings of the
National Academy of Sciences of the United States of America} 112 (1):
184--89. \url{https://www.jstor.org/stable/26460375}.

\hypertarget{ref-Bourne2006}{}
Bourne, Philip E., and Leo M. Chalupa. 2006. ``Ten simple rules for
getting grants.'' \emph{PLoS Computational Biology} 2 (2): 59--60.
doi:\href{https://doi.org/10.1371/journal.pcbi.0020012}{10.1371/journal.pcbi.0020012}.

\hypertarget{ref-Burkle2013}{}
Burkle, Laura A., John C. Marlin, and Tiffany M Knight. 2013.
``Plant-Pollinator Interactions over 120 Years: Loss of Species,
Co-Occurrence, and Function.'' \emph{Science} 339 (March): 1611--6.

\hypertarget{ref-Busch2017}{}
Busch, Jonah, and Kalifi Ferretti-Gallon. 2017. ``What drives
deforestation and what stops it? A meta-analysis.'' \emph{Review of
Environmental Economics and Policy} 11 (1): 3--23.
doi:\href{https://doi.org/10.1093/reep/rew013}{10.1093/reep/rew013}.

\hypertarget{ref-Cash2003}{}
Cash, D.W., W.C. Clark, F. Alcock, N.M. Dickson, N. Eckley, D.H. Guston,
J. Jager, and R.B. Mitchell. 2003. ``Knowledge systems for sustainable
development.'' \emph{Proceedings of the National Academy of Sciences of
the United States of America} 100 (14): 8086--91.
doi:\href{https://doi.org/10.1063/1.4795010}{10.1063/1.4795010}.

\hypertarget{ref-Chiabai2018}{}
Chiabai, Aline, Sonia Quiroga, Pablo Martinez-Juarez, Sahran Higgins,
and Tim Taylor. 2018. ``The nexus between climate change, ecosystem
services and human health: Towards a conceptual framework.''
\emph{Science of the Total Environment} 635. The Authors: 1191--1204.
doi:\href{https://doi.org/10.1016/j.scitotenv.2018.03.323}{10.1016/j.scitotenv.2018.03.323}.

\hypertarget{ref-Colvin2016}{}
Colvin, R. M., G. Bradd Witt, and Justine Lacey. 2016. ``Approaches to
identifying stakeholders in environmental management: Insights from
practitioners to go beyond the 'usual suspects'.'' \emph{Land Use
Policy} 52. Elsevier Ltd: 266--76.
doi:\href{https://doi.org/10.1016/j.landusepol.2015.12.032}{10.1016/j.landusepol.2015.12.032}.

\hypertarget{ref-Davidson2013}{}
Davidson, Marc D. 2013. ``On the relation between ecosystem services,
intrinsic value, existence value and economic valuation.''
\emph{Ecological Economics} 95. Elsevier B.V.: 171--77.
doi:\href{https://doi.org/10.1016/j.ecolecon.2013.09.002}{10.1016/j.ecolecon.2013.09.002}.

\hypertarget{ref-Dickman2010}{}
Dickman, A. J. 2010. ``Complexities of conflict: The importance of
considering social factors for effectively resolving human-wildlife
conflict.'' \emph{Animal Conservation} 13 (5): 458--66.
doi:\href{https://doi.org/10.1111/j.1469-1795.2010.00368.x}{10.1111/j.1469-1795.2010.00368.x}.

\hypertarget{ref-Dickman2014}{}
Dickman, Amy J., Leela Hazzah, Chris Carbone, and Sarah M. Durant. 2014.
``Carnivores, culture and 'contagious conflict': Multiple factors
influence perceived problems with carnivores in Tanzania's Ruaha
landscape.'' \emph{Biological Conservation} 178. Elsevier Ltd: 19--27.
doi:\href{https://doi.org/10.1016/j.biocon.2014.07.011}{10.1016/j.biocon.2014.07.011}.

\hypertarget{ref-Ferguson2015}{}
Ferguson, Peter. 2015. ``The Green Economy Agenda: Business as Usual or
Transformational Discourse?'' Journal Article. \emph{Environmental
Politics} 24 (1): 17--37.
doi:\href{https://doi.org/10.1080/09644016.2014.919748}{10.1080/09644016.2014.919748}.

\hypertarget{ref-Fiutem1996}{}
Fiutem, T., and M. Antoniol. n.d. ``A Cliche-Based Environment to
Support Architectural Reverse Engineering.'' Conference Proceedings. In
\emph{1996 Proceedings of International Conference on Software
Maintenance}, 319--28.
doi:\href{https://doi.org/10.1109/ICSM.1996.565035}{10.1109/ICSM.1996.565035}.

\hypertarget{ref-Fu2018}{}
Fu, Qi, Ying Hou, Bo Wang, Xu Bi, Bo Li, and Xinshi Zhang. 2018.
``Scenario analysis of ecosystem service changes and interactions in a
mountain-oasis-desert system: a case study in Altay Prefecture, China.''
\emph{Scientific Reports} 8 (1): 1--13.
doi:\href{https://doi.org/10.1038/s41598-018-31043-y}{10.1038/s41598-018-31043-y}.

\hypertarget{ref-Gerstner2017}{}
Gerstner, Katharina, David Moreno-Mateos, Jessica Gurevitch, Michael
Beckmann, Stephan Kamback, Holly P. Jones, and Ralf Seppelt. 2017.
``Will your paper be used in a meta-analysis? Make the reach of your
research broader and longer lasting.'' \emph{Methods in Ecology and
Evolution} 8: 777--84.
doi:\href{https://doi.org/10.1111/2041-210X.12758}{10.1111/2041-210X.12758}.

\hypertarget{ref-Halpern2008}{}
Halpern, B.S., S. Walbridge, K.A. Selkoe, C.V. Kappel, F. Micheli, C.
D'Argrosa, J.F. Bruno, et al. 2008. ``A Global Map of Human Impact on
Marine Ecosystems.'' \emph{Science} 319 (February): 948--53.
doi:\href{https://doi.org/10.1111/2041-210X.12109}{10.1111/2041-210X.12109}.

\hypertarget{ref-Hampton2013}{}
Hampton, S.E., C.A. Strasser, J.J. Tewksbury, W.K. Gram, A.E. Budden,
A.L. Batcheller, C.S. Duke, and J.H. Porter. 2013. ``Big Data and the
Future of Ecology.'' Journal Article. \emph{Frontiers in Ecology \& the
Environment} 11: 156--62.

\hypertarget{ref-Hao2018}{}
Hao, Jing. 2018. ``Reconsidering cause inside the clause in scientific
discourse-from a discourse semantic perspective in systemic functional
linguistics.'' \emph{Text and Talk} 38 (5): 525--50.
doi:\href{https://doi.org/10.1515/text-2018-0013}{10.1515/text-2018-0013}.

\hypertarget{ref-Iacona2018}{}
Iacona, Gwenllian D., William J. Sutherland, Bonnie Mappin, Vanessa M.
Adams, Paul R. Armsworth, Tim Coleshaw, Carly Cook, et al. 2018.
``Standardized reporting of the costs of management interventions for
biodiversity conservation.'' \emph{Conservation Biology} 32 (5):
979--88.
doi:\href{https://doi.org/10.1111/cobi.13195}{10.1111/cobi.13195}.

\hypertarget{ref-Johnson2006}{}
Johnson, Arlyne, C. Vongkhamheng, M. Hedemark, and T. Saithongdam. 2006.
``Effects of human-carnivore conflict on tiger (Panthera tigris) and
prey populations in Lao PDR.'' \emph{Animal Conservation} 9 (4):
421--30.
doi:\href{https://doi.org/10.1111/j.1469-1795.2006.00049.x}{10.1111/j.1469-1795.2006.00049.x}.

\hypertarget{ref-Jones2009}{}
Jones, Peter G., and Philip K. Thornton. 2009. ``Croppers to livestock
keepers: livelihood transitions to 2050 in Africa due to climate
change.'' \emph{Environmental Science and Policy} 12 (4): 427--37.
doi:\href{https://doi.org/10.1016/j.envsci.2008.08.006}{10.1016/j.envsci.2008.08.006}.

\hypertarget{ref-Karttunen2018}{}
Karttunen, Kalle, Anssi Ahtikoski, Susanna Kujala, Hannu Törmä, Jouko
Kinnunen, Hannu Salminen, Saija Huuskonen, et al. 2018. ``Regional
socio-economic impacts of intensive forest management, a CGE approach.''
\emph{Biomass and Bioenergy} 118 (July). Elsevier Ltd: 8--15.
doi:\href{https://doi.org/10.1016/j.biombioe.2018.07.024}{10.1016/j.biombioe.2018.07.024}.

\hypertarget{ref-Kelly2006}{}
Kelly, Clint. 2006. ``Replicating Empirical Research in Behavioral
Ecology: How and Why It Should Be Done but Rarely Ever Is.'' Journal
Article. \emph{The Quarterly Review of Biology} 81: 221--36.

\hypertarget{ref-Koontz2018}{}
Koontz, Tomas M., and Craig W. Thomas. 2018. ``Use of science in
collaborative environmental management: Evidence from local watershed
partnerships in the Puget Sound.'' \emph{Environmental Science and
Policy} 88 (June). Elsevier: 17--23.
doi:\href{https://doi.org/10.1016/j.envsci.2018.06.007}{10.1016/j.envsci.2018.06.007}.

\hypertarget{ref-Kotiaho2002}{}
Kotiaho, J.S., and J.L. Tomkins. 2002. ``Meta-analysis, can it ever
fail?'' \emph{Oikos} 96 (3): 551--53.

\hypertarget{ref-Lewinsohn2014}{}
Lewinsohn, Thomas M., José Luiz Attayde, Carlos Roberto Fonseca, Gislene
Ganade, Leonardo Ré Jorge, Johannes Kollmann, Gerhard E. Overbeck, et
al. 2014. ``Ecological literacy and beyond: Problem-based learning for
future professionals.'' \emph{Ambio} 44 (2): 154--62.
doi:\href{https://doi.org/10.1007/s13280-014-0539-2}{10.1007/s13280-014-0539-2}.

\hypertarget{ref-Lindsey2009}{}
Lindsey, P. A., S. S. Romañach, and H. T. Davies-Mostert. 2009. ``The
importance of conservancies for enhancing the value of game ranch land
for large mammal conservation in southern Africa.'' \emph{Journal of
Zoology} 277 (2): 99--105.
doi:\href{https://doi.org/10.1111/j.1469-7998.2008.00529.x}{10.1111/j.1469-7998.2008.00529.x}.

\hypertarget{ref-Lortie2014}{}
Lortie, Christopher J. 2014. ``Formalized synthesis opportunities for
ecology: Systematic reviews and meta-analyses.'' \emph{Oikos} 123 (8):
897--902.
doi:\href{https://doi.org/10.1111/j.1600-0706.2013.00970.x}{10.1111/j.1600-0706.2013.00970.x}.

\hypertarget{ref-Lortie2018}{}
Lortie, Christopher J., A. Filazzola, R. Kelsey, Abigail K. Hart, and H.
S. Butterfield. 2018. ``Better late than never: a synthesis of strategic
land retirement and restoration in California.'' \emph{Ecosphere} 9 (8):
e02367. doi:\href{https://doi.org/10.1002/ecs2.2367}{10.1002/ecs2.2367}.

\hypertarget{ref-Macpherson2017}{}
Macpherson, Ignacio, and Ignacio Segarra. 2017. ``Commentary: Grand
Challenge: ELSI in a Changing Global Environment.'' Journal Article.
\emph{Frontiers in Genetics} 8: 135.
\url{https://www.frontiersin.org/article/10.3389/fgene.2017.00135}.

\hypertarget{ref-McKinley2017}{}
McKinley, Duncan C., Abe J. Miller-Rushing, Heidi L. Ballard, Rick
Bonney, Hutch Brown, Susan C. Cook-Patton, Daniel M. Evans, et al. 2017.
``Citizen science can improve conservation science, natural resource
management, and environmental protection.'' \emph{Biological
Conservation} 208. Elsevier Ltd: 15--28.
doi:\href{https://doi.org/10.1016/j.biocon.2016.05.015}{10.1016/j.biocon.2016.05.015}.

\hypertarget{ref-McKinnon2015}{}
McKinnon, Madeleine. 2015. ``Sustainability: Map the evidence.''
\emph{Nature} 528 (7581): 185--87.
doi:\href{https://doi.org/10.1017/CBO9781107415324.004}{10.1017/CBO9781107415324.004}.

\hypertarget{ref-Miguel2017}{}
Miguel, Thiago Barros, José Max Barbosa Oliveira-Junior, Raphael
Ligeiro, and Leandro Juen. 2017. ``Odonata (Insecta) as a tool for the
biomonitoring of environmental quality.'' \emph{Ecological Indicators}
81 (March). Elsevier: 555--66.
doi:\href{https://doi.org/10.1016/j.ecolind.2017.06.010}{10.1016/j.ecolind.2017.06.010}.

\hypertarget{ref-Mosnier2017}{}
Mosnier, Claire, Anne Duclos, Jacques Agabriel, and Armelle Gac. 2017.
``What prospective scenarios for 2035 will be compatible with reduced
impact of French beef and dairy farm on climate change?''
\emph{Agricultural Systems} 157 (August). Elsevier: 193--201.
doi:\href{https://doi.org/10.1016/j.agsy.2017.07.006}{10.1016/j.agsy.2017.07.006}.

\hypertarget{ref-Naidoo2006}{}
Naidoo, Robin, Andrew Balmford, Paul J. Ferraro, Stephen Polasky, Taylor
H. Ricketts, and Mathieu Rouget. 2006. ``Integrating economic costs into
conservation planning.'' \emph{Trends in Ecology and Evolution} 21 (12):
681--87.
doi:\href{https://doi.org/10.1016/j.tree.2006.10.003}{10.1016/j.tree.2006.10.003}.

\hypertarget{ref-Nielsen2017}{}
Nielsen, Jacqueline A., Eva Grøndahl, Ragan M. Callaway, Katharine J.M.
Dickinson, and Bodil K. Ehlers. 2017. ``Home and away: biogeographical
comparison of species diversity in Thymus vulgaris communities.''
\emph{Biological Invasions} 19 (9). Springer International Publishing:
2533--42.
doi:\href{https://doi.org/10.1007/s10530-017-1461-x}{10.1007/s10530-017-1461-x}.

\hypertarget{ref-Pace2010}{}
Pace, Michael L., Stephanie E. Hampton, Karin E. Limburg, Elena M.
Bennett, Elizabeth M. Cook, Ann E. Davis, J. Morgan Grove, et al. 2010.
``Communicating with the public: Opportunities and rewards for
individual ecologists.'' \emph{Frontiers in Ecology and the Environment}
8 (6): 292--98.
doi:\href{https://doi.org/10.1890/090168}{10.1890/090168}.

\hypertarget{ref-Reed2008}{}
Reed, Mark S. 2008. ``Stakeholder participation for environmental
management: A literature review.'' \emph{Biological Conservation} 141
(10): 2417--31.
doi:\href{https://doi.org/10.1016/j.biocon.2008.07.014}{10.1016/j.biocon.2008.07.014}.

\hypertarget{ref-Roy-Byrne2010}{}
Roy-Byrne, Peter, Michelle G. Craske, Greer Sullivan, Raphael D. Rose,
Mark J. Edlund, Ariel J. Lang, Alexander Bystritsky, et al. 2010.
``Delivery of evidence-based treatment for multiple anxiety disorders in
primary care: A randomized controlled trial.'' \emph{JAMA - Journal of
the American Medical Association} 303 (19): 1921--8.
doi:\href{https://doi.org/10.1001/jama.2010.608}{10.1001/jama.2010.608}.

\hypertarget{ref-Ruxton2017}{}
Ruxton, G.D., and N. Colgrave. 2017. \emph{Experimental Design for the
Life Sciences}. Book. Second. Oxford, UK: Oxford University Press.

\hypertarget{ref-Sandel2015}{}
Sandel, Brody. 2015. ``Towards a taxonomy of spatial scale-dependence.''
\emph{Ecography} 38 (4): 358--69.
doi:\href{https://doi.org/10.1111/ecog.01034}{10.1111/ecog.01034}.

\hypertarget{ref-Tilman2018}{}
Tilman, Andrew R., Simon Levin, and James R. Watson. 2018.
``Revenue-sharing clubs provide economic insurance and incentives for
sustainability in common-pool resource systems.'' \emph{Journal of
Theoretical Biology} 454. Elsevier Ltd: 205--14.
doi:\href{https://doi.org/10.1016/j.jtbi.2018.06.003}{10.1016/j.jtbi.2018.06.003}.

\hypertarget{ref-Towns2009}{}
Towns, Lindsay, A. E. Derocher, I. Stirling, N. J. Lunn, and D. Hedman.
2009. ``Spatial and temporal patterns of problem polar bears in
Churchill, Manitoba.'' \emph{Polar Biology} 32 (10): 1529--37.
doi:\href{https://doi.org/10.1007/s00300-009-0653-y}{10.1007/s00300-009-0653-y}.

\hypertarget{ref-Tranfield2003}{}
Tranfield, David, David Denyer, and Palminder Smart. 2003. ``Towards a
Methodology for Developing Evidence-Informed Management Knowledge by
Means of Systematic Review'' 14: 207--22.

\hypertarget{ref-Vogt2018}{}
Vogt, R.J., S. Sharma, and P.R. Leavitt. 2018. ``Direct and interactive
effects of climate, meteorology, river hydrology, and lake
characteristics on water quality in productive lakes of the Canadian
Prairies.'' \emph{Canadian Journal of Fisheries and Aquatic Sciences} 75
(1): 47--59.
doi:\href{https://doi.org/10.1139/cjfas-2016-0520}{10.1139/cjfas-2016-0520}.

\hypertarget{ref-Zhu2010}{}
Zhu, Lizhong, Li Lu, and Dong Zhang. 2010. ``Mitigation and Remediation
Technologies for Organic Contaminated Soils.'' Journal Article.
\emph{Frontiers of Environmental Science \& Engineering in China} 4 (4):
373--86.
doi:\href{https://doi.org/10.1007/s11783-010-0253-7}{10.1007/s11783-010-0253-7}.



\end{document}
