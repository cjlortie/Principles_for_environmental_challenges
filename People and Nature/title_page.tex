%% Submissions for peer-review must enable line-numbering
%% using the lineno option in the \documentclass command.
%%
%% Preprints and camera-ready submissions do not need
%% line numbers, and should have this option removed.
%%
%% Please note that the line numbering option requires
%% version 1.1 or newer of the wlpeerj.cls file, and
%% the corresponding author info requires v1.2

\documentclass[fleqn,10pt]{wlpeerj} % for preprint submissions

% ZNK -- Adding headers for pandoc

\setlength{\emergencystretch}{3em}
\providecommand{\tightlist}{
\setlength{\itemsep}{0pt}\setlength{\parskip}{0pt}}
\usepackage{lipsum}
\usepackage[unicode=true]{hyperref}
\usepackage{longtable}



\usepackage{lipsum}

\title{Simple principles for engineering reproducible solutions to
environmental management challenges.}

\author[1]{Christopher J. Lortie}

\corrauthor[1]{Christopher J. Lortie}{\href{mailto:lortie@yorku.ca}{\nolinkurl{lortie@yorku.ca}}}
\author[1]{Malory Owen}


\affil[1]{Biology, York University, 4700 Keele St.~Toronto, ON, Canada, M3J1P3}


%
% \author[1]{First Author}
% \author[2]{Second Author}
% \affil[1]{Address of first author}
% \affil[2]{Address of second author}
% \corrauthor[1]{First Author}{f.author@email.com}

% 

\begin{abstract}
An environmental management challenge is an opportunity to use
fundamental science to inform evidence-based decisions for environmental
stakeholders and conservationists. Contemporary science is embracing
open science and increasingly conscious of reproduciblility.
Synergistically, applying these two paradigms in concert advances our
capacity to move beyond context dependency and singlular thinking to
reverse engineer solutions from published scientific evidence associated
with one challenge to many. Herein, we provide a short list of
principles that can guide those that seek solutions to address
environmental management through primary scientific literature.
% Dummy abstract text. Dummy abstract text. Dummy abstract text. Dummy abstract text. Dummy abstract text. Dummy abstract text. Dummy abstract text. Dummy abstract text. Dummy abstract text. Dummy abstract text. Dummy abstract text.
\end{abstract}

\begin{document}

\flushbottom
\maketitle
\thispagestyle{empty}

\hypertarget{article-type}{%
\section*{Article type}\label{article-type}}
\addcontentsline{toc}{section}{Article type}

Perspective.

\hypertarget{acknowledgments}{%
\section*{Acknowledgments}\label{acknowledgments}}
\addcontentsline{toc}{section}{Acknowledgments}

This synthesis was supported by NSERC DG to CJL, and FGS funding to MO
from York University.

\hypertarget{author-biographies}{%
\section*{Author Biographies}\label{author-biographies}}
\addcontentsline{toc}{section}{Author Biographies}

Christopher J. Lortie is a professor of integrative ecology at York
University in Canada and a Senior Research Fellow at the National Center
for Ecological Analysis and Synthesis at The University of California
Santa Barbara. Malory Owen is an ecologist and environmental scientist
at York University.

\hypertarget{data-archiving}{%
\section*{Data archiving}\label{data-archiving}}
\addcontentsline{toc}{section}{Data archiving}

There are no data to archive.



\end{document}
